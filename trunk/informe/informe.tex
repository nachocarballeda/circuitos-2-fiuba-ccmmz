\documentclass[12pt,a4paper]{article}
\usepackage[utf8]{inputenc}
\usepackage[spanish]{babel}
\usepackage{amsmath}
\usepackage{amsfonts}
\usepackage{amssymb}
\usepackage{graphicx}
\usepackage{fancyhdr}
\usepackage{geometry}
\geometry{a4paper,tmargin=1in,bmargin=1in,lmargin=0.6in,rmargin=0.8in} 


% Símbolos de las unidades, sobre todo es para que en una ecuacion no queden en cursiva

\newcommand{\volt}{\mbox{V}}
\newcommand{\mvolt}{\mbox{mV}}
\newcommand{\hertz}{\mbox{Hz}}
\newcommand{\khertz}{\mbox{kHz}}
\newcommand{\Mhertz}{\mbox{MHz}}
\newcommand{\farad}{\mbox{F}}
\newcommand{\nfarad}{\mbox{nF}}
\newcommand{\pfarad}{\mbox{pF}}
\newcommand{\ohm}{\Omega}
\newcommand{\kohm}{\mbox{k}\Omega}
\newcommand{\Mohm}{\mbox{M}\Omega}
\newcommand{\amper}{\mbox{A}}
\newcommand{\mamper}{\mbox{mA}}
\newcommand{\segundo}{\mbox{s}}
\newcommand{\msec}{\mbox{ms}}
\newcommand{\usec}{\mu\mbox{s}}
\newcommand{\nsec}{\mbox{ns}}
\newcommand{\kwatt}{\mbox{kW}}

\providecommand{\E}[1]{\ensuremath{\times 10^{#1}}}
%para que al ir escribiendo se pueda poner \E{potencia} y quede:" x 10^potencia "

%------------------------- Inicio del documento ---------------------------

\begin{document}

%
% Sin cabecera ni pie de página:
%
\thispagestyle{empty}

\begin{figure}[t]
    \centering
    \includegraphics [scale=0.72]{img/logo_fiuba_alta.jpg}
\end{figure}

\vspace{5.5cm}

\begin{center}
    \Large{Departamento de Electrónica}\\
    \huge{66.10 Circuitos Electrónicos II}\\
    \vspace{.5cm}
    \large{Proyecto: Amplificador de Audio de Potencia Clase G}\\
    \vspace{1cm}
    \begin{tabular}{lc}
    \textbf{Chaure Fernando} & 90389 \\
    \textbf{Combier Natasha} & Intercambio \\    
    \textbf{Marchi Pablo} & 90603 \\
    \textbf{Müller Miguel} & 86130 \\
    \textbf{Zurita Francisco} & 89722 \\ 
    \end{tabular}\\
    \vspace{.3cm}
    \small{\today}\\
\end{center}

\vspace{1.5cm}

\begin{center}
\begin{tabular}{|c|c|c|c|}
\hline
Cuatrimestre / Año & 1\sptext{er} cuatrimestre 2012 \\
\hline
Profesores: & Ing. Alberto Bertuccio \\

\hline
\end{tabular}


\vspace{1cm}
\begin{tabular}{|c|c|}
\hline
Fecha de entrega & Firma  \\
\hline
\hspace*{1cm} & \hspace*{3cm}\\
\hline

\hline
\end{tabular}

\vspace{.5cm}

\begin{tabular}{|c|c|c|c|c|}
\hline
\,Nota\, &\multicolumn{3}{|c|}{Fecha de aprobación} & Firma  \\
\hline
\hspace*{1.7cm} & \hspace*{1cm}& \hspace*{1cm} & \hspace*{1cm}  & \hspace*{3cm}\\
\hline
\end{tabular}

\vspace{1cm}
Obsevaciones:
\hrulefill\par
\vspace{.3cm}
\hrulefill\par
\vspace{.3cm}
\hrulefill\par
\vspace{.3cm}
\hrulefill\par
\vspace{.3cm}
\hrulefill\par
\vspace{.3cm}


\end{center}


\newpage
\thispagestyle{empty}
\tableofcontents
\newpage

%estilo de pie y encabezado de pagina
\pagestyle{fancy}
\lhead{66.10 Circuitos Electrónicos 2}
%\rhead{Exploración preeliminar de circuitos amplificadores}
%\lfoot{Albani - González Zerbo - Piccinini}
\rfoot{1\sptext{er} cuatrimestre 2012}


%\input{cada parte del tp}


\end{document}

