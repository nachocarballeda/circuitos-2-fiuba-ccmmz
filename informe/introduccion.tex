\section{Introducción}
\bigskip

El presente informe detalla el diseño e implementación de un amplificador de audio clase G. En la realización de este proyecto han sido volcados los conocimientos de la materia Circuitos Electrónicos II. En la Figura~\ref{esquema_bloques}, se muestra el diagrama en bloques de las partes fundamentaes del proyecto.
	
	\begin{figure}[H]
	\centering
	\includegraphics[scale=0.65]{img/esquema_bloques.png}
	\caption{Esquema en bloques.}
	\label{esquema_bloques} 
	\end{figure}
	
\medskip 
\subsection{Preamplificador}
	
Un preamplificador es un circuito que permite adaptar las diferentes señales de entrada para luego poder ingresarlas a una etapa de potencia. Este circuito puede servir para adaptar señales de diferentes fuentes, por ejemplo: micrófonos, reproductores de mp3, salidas de placas de sonido de  pc, etc. Como todos estos dispositivos no tienen el mismo nivel de salida, el preamplificador es quien se encarga de llevar a todas estas señales a una tensión de estipulada que luego entra a la etapa de potencia anteriormente nombrada. Los preamplificadores suelen ser de baja potencia y de realizarse de forma adecuada no deben distorsionar en gran medida la señal.

Alguno de los controles que pueden tener los preamplificadores son:
	
\begin{itemize}
	\item Control de volumen
	\item Control de tono
	\item Control de balance
	\item Selector de canal de entrada 
	\item Amplificación
	\end{itemize}	
	
	\medskip 
\subsection{Amplificadores de Potencia}
	
Un amplificador debe satisfacer ciertos requerimientos especiales. Uno de los más importantes es el de entregar una señal con una cantidad específica de potencia a una carga con niveles aceptablemente bajos de distorsión. Otro objetivo común en el diseño es minimizar la impedancia de salida, de tal forma que la ganancia de voltaje quede relativamente poco afectada por el valor de la impedancia de carga. Una etapa de salida bien diseñada debe cumplir con estas características de funcionamiento, consumiendo poca potencia en estado de reposo, sin que esto represente una limitación importante en la respuesta en frecuencia del amplificador. 
 
Los amplificadores de potencia  se clasifican generalmente en seis tipos: A, B, AB , C y G para diseños analógicos y clases D y E para los diseños de conmutación. 
\medskip 


\subsubsection*{Amplificadores Clase A}


En esta clase de amplificadores se usa un solo transistor. El emisor seguidor es la etapa de salida clase A mas utilizada. La corriente de salida circula durante todo el ciclo de la señal de entrada, ya que el transistor esta polarizado con una corriente continua. Esta es una de las grandes desventajas de este tipo de amplificador ya que consume potencia en ausencia de señal y por lo tanto es lógico esperar un rendimiento pobre que en general no supera el 25\%. Como ventaja la distorsión introducida suele ser baja. En la Figura~\ref{ampliA} se muestra un ejemplo de este tipo de amplificador.
 
\begin{figure}[H]
\centering
\includegraphics[scale=0.6]{img/ampliA.png}
\caption{Ejemplo, amplificador clase A}
\label{ampliA} 
\end{figure}

\medskip 
\subsubsection*{Amplificador Clase B}

Esta clase de amplificadores se compone de un par de transistores (uno pnp y otro npn) conectados de forma tal que no se encuentren ambos en la zona de modo activo directo en el mismo instante de tiempo. Es decir, si suponemos tener una entrada senoidal, durante un semiciclo uno de los transistores se encuentra en la región activa, conduciendo corriente, mientras que el otro se encuentra en corte y durante el otro semiciclo viceversa.
 Una ventaja de esta amplificador sobre la clase A, es que los transistores no disipan potencia en ausencia de señal, lo cual mejora la vida util de los transistores y el rendimiento notablemente, alcanzando un máximo del 78\%.
 La desventaja en este tipo de amplificadores es la llamada “distorsión por cruce”. Es fácil detectar su procedencia al analizar la Figura~\ref{ampliB}.

\begin{figure}[H]
\centering
\includegraphics[scale=0.8]{img/ampliB.png}
\caption{Ejemplo, salida clase B.}
\label{ampliB} 
\end{figure}


Se observa que hay un intervalo de tensiones en el cual los transistores no conducen, ese rango generalmente esta dado por $\pm$0.7 V y esta dado por las curvas características de transferencia.

\medskip 
\subsubsection*{Amplificador Clase “AB”}


 Este tipo de amplificadores recurre a la misma topología utilizada en la etapa de salida de los amplificadores clase B, con la salvedad de que aquí en los transistores circulan una corriente de polarización a modo de reducir notablemente la “distorsión por cruce”.
 Existen diferentes formas de logra dicho tipo de polarización. Las mas sencillas implican agregar un resistor o diodos, por los que circula una corriente fija dada por el circuito de polarización o fuente de corriente. La otra forma es utilizar los circuitos conocidos como multiplicadores de VBE , que resulta ser la forma empleada en este trabajo práctico.

\medskip 
\subsubsection*{Amplificador Clase C}


La corriente de salida solo circula durante menos de medio ciclo de la señal de entrada. Y luego se complementa la salida con un circuito compuesto de capacitores e inductores.
La clase C trabaja para una banda de frecuencias estrecha y resulta muy apropiado en equipos de radiofrecuencia. Esto es debido al fenómeno de resonancia el cual se genera a la salida del amplificador cuando es sintonizado (la impedancia capacitiva e inductiva se cancelan a una frecuencia previamente calculada), aunque no trabaja arriba de 180 grados de ciclo, este amplificador a la salida genera una señal de ciclo completo de señal para la frecuencia fundamental. En la Figura~\ref{ampliC} se muestra un ejemplo de una amplificador de esta clase.
No se utiliza en sonido, por su gran nivel de distorsión y por que su operación no esta destinada para amplificadores de gran señal o gran potencia.

\begin{figure}[H]
\centering
\includegraphics[scale=0.35]{img/ampliC.png}
\caption{Ejemplo, amplificador clase C.}
\label{ampliC} 
\end{figure}

\medskip 
\subsubsection*{Amplificador Clase D}

Esta clase de amplificadores usa señales de pulso (digitales). El uso de técnicas digitales hace posible obtener una señal que varía a lo largo del ciclo completo para producir la salida a partir de muchas partes de la señal de entrada. La principal ventaja de la operación en clase D es que los transistores MOSFET de salida trabajan solo en corte y saturación por lo que teóricamente no se disipa potencia en forma de calor y la eficiencia general puede ser muy alta, de entre 90\% a 99\%. En la practica los MOSFETS solo disipan potencia cuando se encuentran conduciendo (saturación) debido a la pequeña resistencia de encendido que poseen, llamada $R_dson$, de todas maneras esta potencia es despreciable ya que $R_dson$ es del orden de las milésimas de ohm. Se utilizan transistores MOSFET ya que son los únicos capaces de conmutar a las elevadas frecuencias de trabajo, del orden de las centenas de KHz llegando a los MHz en algunos casos.

\medskip 
\subsubsection*{Amplificadores Clase G}


Un amplificador clase G funciona conmutando fuentes de alimentación. Para analizar su funcionamiento tendremos en cuenta un circuito básico como se muestra en la Figura~\ref{ampliG}. Mientras el nivel de la señal de entrada sea pequeño (dentro del margen de +/- V1), el amplificador toma la potencia de la fuente V1. Si la señal de entrada excede el nivel de tensión dado por V1, el circuito automáticamente corta el suministro dado por V1 y conmuta a la fuente de alimentación V2 como puede verse en la Figura~\ref{ampliG_salida}. De esta forma la disipación de potencia es compartida por los transistores de salida, logrando así una menor disipación de potencia y una mayor eficiencia.
En la práctica, la clase G se considera linealmente pobre, comparada con la clase B, dado que la conmutación de las fuentes de alimentación se realiza mediante unos diodos, dando de esta manera un resultado alineal, ya que los mismos deben almacenar y desalojar cargas.
 
\begin{figure}[H]
 \centering
 \includegraphics[scale=0.55]{img/ampliG.png}
 \caption{Ejemplo, amplificador clase G.}
 \label{ampliG} 
 \end{figure}
  
\begin{figure}[H]
 \centering
 \includegraphics[scale=0.55]{img/ampliG_salida.png}
 \caption{Encendido de fuentes V2 en salidas clase G.}
 \label{ampliG_salida} 
 \end{figure}

\subsection{Principales Especificaciones de un Amplificador}
\medskip 
\subsubsection*{Potencia Máxima}

Potencia máxima eficaz, o potencia media a régimen continuo es la potencia eléctrica real verificable con instrumentos que puede proporcionar la etapa de salida  a una frecuencia de 1 kHz (frecuencias medias) sobre la impedancia nominal especificada por el fabricante (normalmente 4$\ohm$, 6$\ohm$ u $8\ohm$) y viene dada por la expresión $P_O=  \frac{V_{O(rms)}^2}{Z_O}$. Donde:
\begin{description}
\item $P_O$ es la potencia de salida
\item $V_{O(rms)}$ es la tensión eficaz de salida
\item $Z_O$ es la impedancia de salida
\end{description}

Se especifica la potencia máxima del amplificador en función de una determinada impedancia, generalmente $8\ohm$. Por ejemplo: 100 WRMS sobre 8$\ohm$.
Cabe destacar que si el amplificador es estéreo hay que tener en cuenta si la potencia se refiere a ambos o a cada uno de los canales.
\medskip 
\subsubsection*{Respuesta en Frecuencia}

Es un rango de frecuencias dentro del cual el amplificador responde de igual forma (respuesta plana). Este rango se espera que como mínimo incluya las audiofrecuencias ( 20 a 20kHz).
Pueden especificarse las frecuencias de corte, en donde la potencia cae a la mitad o la tensión de salida cae en 3dB o sino un rango de frecuencias en donde se cumple que la variación en la tensión de salida no supera una cota dado por el fabricante.
\medskip 
\subsubsection*{Rango Dinámico}

El rango dinámico(DR) es el conjunto de valores entre los niveles de mayor y menor salida, en donde el amplificador reproduce fielmente. En general viene especificado en decibeles y en donde el límite superior esta acotado por la distorsión mientras que el menor esta restringido por el ruido de salida. El rango dinámico se calcula con la relación entre ambos limites, de la siguiente forma:

\begin{equation}\label{rango_dinamico_eq}
DR= \frac{S+N}{N}
\end{equation}

donde:
\begin{description}
\item S es la señal máxima permitida
\item N es la señal de ruido
\item DR es el rango dinámico
\end{description}
\medskip 
\subsubsection*{Distorsión Armónica Total}

Si en un sistema no lineal introducimos un tono de frecuencia $f_0$, en la salida tendremos ese mismo tono (con una amplitud y fase posiblemente diferentes) y, sumado a el, otros tonos de frecuencia $2f_0, 3f_0, ...$ llamados armónicos del tono fundamental . Por lo tanto la THD se calcula de la siguiente forma:

\begin{equation}\label{THD_eq}
THD= \frac{\sum Potencia~de ~los ~armonicos}{Potencia~ de ~la ~frecuencia fundamental}=\frac{P_0+P_1+...+P_N}{P_0}
\end{equation}

Es decir, la distorsión armónica es el valor rms de componentes armónicos de la señal de salida, expresadas como un porcentaje rms del fundamental.
Visto de otra forma, la distorsión describe la variación de la forma de onda de la salida del equipo, con respecto a la señal esperada, si el sistema fuese lineal, con respecto a una determinada entrada y se debe básicamente a la alinealidad de los mismos.
\medskip 
\subsubsection*{Distorsión por Intermodulación}


Es la distorsión que se produce cuando dos o mas señales atraviesan simultáneamente un sistema no lineal. Si dos tonos son reproducidos a la vez, pueden interactuar entre sí en el equipo y producir, asimismo, otros nuevos tonos, que son ni más ni menos que la suma y la diferencia de los dos tonos originales (es lo que se conoce como la frecuencia de batido o pulsaciones). Generalmente, los nuevos tonos no son armónicos entre sí ni con los anteriores debido a que la señal salida no es una combinación lineal de la entrada.
\medskip 
\subsubsection*{Distorsión por Intermodulación Transitoria}


Este tipo de distorsión se da principalmente por el retardo que sufre la señal al ser realimentada negativamente. Todo amplificador demora un tiempo entre que la señal de entrada es aplicada y se obtiene la salida correspondiente, llamado tiempo de tránsito. Es decir, cuando utilizamos una realimentación negativa es esperable que al colocar una entrada inmediatamente obtengamos un efecto de la realimentación que afecte a la misma, pero debido a este tiempo de tránsito aparece un efecto no deseado y por lo tanto este tipo de distorsión. Esta altamente relacionada con el slew rate y con el ancho de banda a lazo abierto del sistema. 
\medskip 
\subsubsection*{Slew Rate}
	
Es la máxima pendiente que puede tener la tensión de salida sin sufrir deformaciones. Generalmente se mide en $\frac{V}{\usec}$ y se calcula como:
\begin{equation}
SR = F(max) \times 2\pi \times V_p
\end{equation}
\begin{description}
\item F(max)= Frecuencia máxima de operación
\item  $V_p$= Tensión pico de onda
\end{description}
\medskip 
\subsubsection*{Sensibilidad}

Este parámetro es una relación entre el valor de tensión de entrada que es necesario para producir la máxima potencia de salida y dicha señal de salida.Por lo general se especifica en decibeles a una determinada impedancia. Si la señal de entrada supera el valor especificado por la sensibilidad no existe ninguna garantía que la señal de salida no sufra un recorte que termine dañando algún componente.
\medskip 
\subsubsection*{Relación Señal a Ruido}

La relación señal/ruido se define como el cociente que existe entre la potencia de la señal que se transmite y la potencia del ruido que la corrompe. Este margen es medido en decibeles. A su vez también es importante definir la figura de ruido. La magnitud del ruido generado por un dispositivo electrónico, por ejemplo un amplificador, se puede expresar mediante la denominada figura de ruido (F), que es el resultado de dividir la relación señal/ruido en la entrada $(S/R)_{entrada}$ por la relación señal/ruido en la salida $(S/R)_{salida}$, cuando los valores de señal y ruido se expresan en números simples :

\begin{equation}
F=\frac{(S/R)_{salida}}{(S/R)_{entrada}}
\end{equation}
\medskip 
\subsubsection*{Impedancia de Entrada}


 Es la impedancia equivalente que vería un generador aplicado a la entrada del amplificador. Para el caso particular de este tipo de amplificador (de tensión) buscamos que sea relativamente alta y no cargue a la etapa anterior. Claramente depende de la frecuencia de operación pero un valor típico para el rango de audiofrecuencias es de 10K $\ohm$.
\medskip 
\subsubsection*{Impedancia de Salida}


Es la impedancia equivalente que vería un generador aplicado a la salida del amplificador. En el caso particular del amplificador de audio buscamos que sea muy baja dado que las cargas son relativamente bajas y de lo contrario nos acortarían la amplitud de la señal de salida. Claramente depende de la frecuencia de operación pero un valor típico para el rango de audiofrecuencias es de décimas o centésimas de $\ohm$.
\medskip 
\subsubsection*{Factor de Amortiguamiento}

Indica la relación entre la impedancia nominal del parlante a conectar y la impedancia de salida del amplificador. Un factor de amortiguamiento alto permite mayor control del movimiento de los 
altavoces (evita oscilaciones) y por tanto reduce la distorsión, especialmente en graves. 
\bigskip
\subsection{Fuentes de Alimentación}

\subsubsection*{Fuentes Lineales}

Este tipo de fuentes tienen un diseño relativamente simple, que puede llegar a ser más complejo cuanto mayor es la corriente que deben suministrar, en lineas generales siguen el esquema de la Figura~\ref{fuente_lineal_tipo}.

\begin{figure}[H]
\centering
\includegraphics[width=0.8\textwidth]{img/fuente_lineal_tipo.png}
\caption{Esquema fuente lineal típica.}
\label{fuente_lineal_tipo} 
\end{figure}

En primer lugar el transformador adapta los niveles de tensión y proporciona aislamiento galvánico. El circuito que convierte la corriente alterna en continua se llama rectificador, luego suelen llevar un circuito que disminuye el rizado. La regulación, o estabilización de la tensión a un valor establecido, se consigue con un componente denominado regulador de tensión. La salida puede ser simplemente un capacitor. 

Las ventajas de las fuentes lineales son una mejor regulación, velocidad y buenas características EMC. Y sus principales desventajas son el bajo rendimiento del rectificador y el tamaño del transformador utilizado. 

\subsubsection*{Fuentes Conmutadas}

Una fuente conmutada es un dispositivo electrónico que transforma energía eléctrica mediante transistores en conmutación. Mientras que un regulador de tensión utiliza transistores polarizados en su región activa de amplificación, las fuentes conmutadas utilizan los mismos conmutándolos activamente a altas frecuencias (20-100 kHz típicamente) entre corte y saturación. La forma de onda cuadrada resultante es aplicada a transformadores con núcleo de ferrita (Los núcleos de hierro no son adecuados para estas altas frecuencias) para obtener uno o varios voltajes de salida de corriente alterna que luego son rectificados (Con diodos rápidos) y filtrados para obtener los voltajes de salida de corriente continua. Las ventajas de este método incluyen menor tamaño y peso del núcleo, mayor eficiencia y por lo tanto menor calentamiento. Las desventajas comparándolas con fuentes lineales es que son mas complejas y generan ruido eléctrico de alta frecuencia que debe ser cuidadosamente minimizado para no causar interferencias a equipos próximos a estas fuentes. La Figura~\ref{fuente_conmutada_esquema_bloq} muestra un esquema en bloques de este tipo de fuente.

\begin{figure}[H]
\centering
\includegraphics[width=0.6\textwidth]{img/fuente_conmutada_esquema.jpg}
\caption{Diagrama en bloques fuente conmutada.}
\label{fuente_conmutada_esquema_bloq} 
\end{figure}


La regulación se obtiene con el conmutador, normalmente un circuito PWM (Pulse Width Modulation) que cambia el ciclo de trabajo. Aquí las funciones del transformador son las mismas que para fuentes lineales pero su posición es diferente. El segundo rectificador convierte la señal alterna pulsante que llega del transformador en un valor continuo. La salida puede ser también un filtro de condensador o uno del tipo LC. Las fuentes conmutadas obtienen un mejor rendimiento, menor coste y tamaño comparadas con las lineales.

Existen diversas topologías para este tipo de fuente, aquí solo se mencionarán algunas.

\paragraph*{Topología Flyback: }
Dada su sencillez y bajo costo, es la topología preferida en la mayoría de los convertidores de baja potencia (hasta 100 w). En la Figura~\ref{topo_flyback} se muestran los principios de esta topología de fuente conmutada. Cuando T conduce, la corriente crece linealmente en el primario del transformador. Cuando T se bloquea, el flujo en el transformador cesa generando una corriente inversa en el secundario que carga el condensador a través del diodo alimentando la carga. El condensador mantiene la tensión en la carga durante el período en que T conduce.

\begin{figure}[H]
\centering
\includegraphics[width=0.25\textwidth]{img/topo_flyback.png}
\caption{Topología Flyback.}
\label{topo_flyback} 
\end{figure}

La regulación de tensión en la salida se obtiene mediante comparación con una referencia fija, actuando sobre el tiempo de encendido del transistor, por tanto la energía transferida a la salida mantiene la tensión constante independientemente del valor de la carga o del valor de la tensión de entrada. Los estados del transistor se controlan por modulación de ancho de pulso (PWM) a frecuencia fija. Como se observa en la Figura~\ref{topo_flyback_mul}, esta topología puede implementarse con múltiples bobinados secundarios de manera tal de proveer de manera independiente varias tensiones.

\begin{figure}[H]
\centering
\includegraphics[width=0.32\textwidth]{img/topo_flyback_mul.png}
\caption{Topología flyback con salidas multiples.}
\label{topo_flyback_mul} 
\end{figure}

\paragraph*{Topología Forward: }
Como se ve en la Figura~\ref{topo_forward} es algo más complejo que el sistema Flyback pero rentable en cuanto a costes para potencias de 100 a 250w.Cuando el transistor está conduciendo, la corriente crece en el primario del transformador transfiriendo energía al secundario. La corriente pasa a través de la inductancia L a la carga, acumulándose energía magnética en L.Cuando T se apaga, la corriente en el primario cesa invirtiendo la tensión en el secundario. En este momento D2 queda polarizado inversamente bloqueando la corriente de secundario, pero D3 conduce permitiendo que la energía almacenada en L se descargue alimentando a la carga. El tercer devanado permite aprovechar la energía que queda en el transformador devolviéndola a la entrada, vía D1.

Contrariamente al método Flyback, la inductancia cede energía a la carga todo el tiempo, esto hace que los diodos soporten la mitad de la corriente y los niveles de rizado de salida sean más bajos.

\begin{figure}[H]
\centering
\includegraphics[width=0.35\textwidth]{img/topo_forward.png}
\caption{Topología Forward.}
\label{topo_forward} 
\end{figure}


\paragraph*{Topología Push-Pull: }

Esta topología mostrada en la Figura~\ref{topo_pull-push},  se desarrolló para aprovechar mejor los núcleos magnéticos. En esencia consisten en dos convertidores Forward controlados por dos entradas en contrafase. Los diodos D1 y D2 en el secundario, actúan como dos diodos de recuperación. Idealmente los períodos de conducción de los transistores deben ser iguales, el transformador se excita simétricamente y al contrario de la topología Forward no es preciso prever entrehierro en el circuito magnético, ya que no existe asimetría en el flujo magnético y por tanto componente continua. Ello se traduce en una reducción del volumen del núcleo del orden del 50\% para una misma potencia.

\begin{figure}[H]
\centering
\includegraphics[width=0.35\textwidth]{img/topo_pull-push.png}
\caption{Topología Push-Pull.}
\label{topo_pull-push} 
\end{figure}



% El desarrollo del trabajo fue encarado como un caso real de la vida profesional, en el cual se nos han dado las especificaciones y basamos en ellas nuestro diseño, tratando de ser lo mas eficientes al menor costo posible y con los productos que se pudieron encontrar en el mercado.
%	
%Durante el desarrollo del trabajo hemos ido encontrando inconvenientes, ya sea errores humanos o diferencias entre las simulaciones y la implementación material. Se detallaron dichos problemas ya que consideramos que contribuyen al proceso de aprendizaje del diseño real.