\documentclass[12pt,a4paper]{article}
\setlength{\headheight}{16pt}
\usepackage[utf8]{inputenc}
\usepackage[spanish]{babel}
\usepackage{amsmath}
\usepackage{amsfonts}
\usepackage{amssymb}
\usepackage{graphicx}
\usepackage{fancyhdr}
\usepackage{float}
\usepackage{geometry}
\geometry{a4paper,left=2cm,right=2cm,top=2cm,bottom=2cm} 
%Para que no recorte las palabras con guiones
\tolerance=1000
\pretolerance=1000
% Símbolos de las unidades, sobre todo es para que en una ecuacion no queden en cursiva

\newcommand{\volt}{\mbox{V}}
\newcommand{\mvolt}{\mbox{mV}}
\newcommand{\hertz}{\mbox{Hz}}
\newcommand{\khertz}{\mbox{kHz}}
\newcommand{\Mhertz}{\mbox{MHz}}
\newcommand{\farad}{\mbox{F}}
\newcommand{\nfarad}{\mbox{nF}}
\newcommand{\pfarad}{\mbox{pF}}
\newcommand{\ohm}{\Omega}
\newcommand{\kohm}{\mbox{k}\Omega}
\newcommand{\Mohm}{\mbox{M}\Omega}
\newcommand{\amper}{\mbox{A}}
\newcommand{\mamper}{\mbox{mA}}
\newcommand{\segundo}{\mbox{s}}
\newcommand{\msec}{\mbox{ms}}
\newcommand{\usec}{\mu\mbox{s}}
\newcommand{\nsec}{\mbox{ns}}
\newcommand{\kwatt}{\mbox{kW}}

\providecommand{\E}[1]{\ensuremath{\times 10^{#1}}}
%para que al ir escribiendo se pueda poner \E{potencia} y quede:" x 10^potencia "

%------------------------- Inicio del documento ---------------------------

\begin{document}

%
% Sin cabecera ni pie de página:
%
\thispagestyle{empty}

\begin{figure}[t]
    \centering
    \includegraphics [scale=0.72]{img/logo_fiuba_alta.jpg}
\end{figure}

\vspace{5.5cm}

\begin{center}
    \Large{Departamento de Electrónica}\\
    \huge{66.10 Circuitos Electrónicos II}\\
    \vspace{.5cm}
    \large{Proyecto: Amplificador de Audio de Potencia Clase G}\\
    \vspace{1cm}
    \begin{tabular}{lc}
    \textbf{Chaure Fernando} & 90389 \\
    \textbf{Combier Natasha} & Intercambio \\    
    \textbf{Marchi Pablo} & 90603 \\
    \textbf{Müller Miguel} & 86130 \\
    \textbf{Zurita Francisco} & 89722 \\ 
    \end{tabular}\\
    \vspace{.3cm}
    \small{\today}\\
\end{center}

\vspace{1.5cm}

\begin{center}
\begin{tabular}{|c|c|c|c|}
\hline
Cuatrimestre / Año & 1\sptext{er} cuatrimestre 2012 \\
\hline
Profesores: & Ing. Alberto Bertuccio \\

\hline
\end{tabular}


\vspace{1cm}
\begin{tabular}{|c|c|}
\hline
Fecha de entrega & Firma  \\
\hline
\hspace*{1cm} & \hspace*{3cm}\\
\hline

\hline
\end{tabular}

\vspace{.5cm}

\begin{tabular}{|c|c|c|c|c|}
\hline
\,Nota\, &\multicolumn{3}{|c|}{Fecha de aprobación} & Firma  \\
\hline
\hspace*{1.7cm} & \hspace*{1cm}& \hspace*{1cm} & \hspace*{1cm}  & \hspace*{3cm}\\
\hline
\end{tabular}

\vspace{1cm}
Obsevaciones:
\hrulefill\par
\vspace{.3cm}
\hrulefill\par
\vspace{.3cm}
\hrulefill\par
\vspace{.3cm}
\hrulefill\par
\vspace{.3cm}
\hrulefill\par
\vspace{.3cm}


\end{center}


\newpage
\thispagestyle{empty}
\tableofcontents
\newpage


%estilo de pie y encabezado de pagina
\pagestyle{fancy}
\lhead{66.10 Circuitos Electrónicos 2}
%para sacar el margen derecho superior
%\fancyhead[R]{}
\fancyfoot[C]{1\sptext{er} cuatrimestre 2012}
\setcounter{page}{1}
\fancyfoot[R]{\thepage}


%Seciones y subseciones del tp

\section{Introducción}
\bigskip

El presente informe detalla el diseño e implementación de un amplificador de audio clase G. En la realización de este proyecto han sido volcados los conocimientos de la materia Circuitos Electrónicos II. En la Figura~\ref{esquema_bloques}, se muestra el diagrama en bloques de las partes fundamentaes del proyecto.
	
	\begin{figure}[H]
	\centering
	\includegraphics[scale=0.65]{img/esquema_bloques.png}
	\caption{Esquema en bloques.}
	\label{esquema_bloques} 
	\end{figure}
	
\medskip 
\subsection{Preamplificador}
	
Un preamplificador es un circuito que permite adaptar las diferentes señales de entrada para luego poder ingresarlas a una etapa de potencia. Este circuito puede servir para adaptar señales de diferentes fuentes, por ejemplo: micrófonos, reproductores de mp3, salidas de placas de sonido de  pc, etc. Como todos estos dispositivos no tienen el mismo nivel de salida, el preamplificador es quien se encarga de llevar a todas estas señales a una tensión de estipulada que luego entra a la etapa de potencia anteriormente nombrada. Los preamplificadores suelen ser de baja potencia y de realizarse de forma adecuada no deben distorsionar en gran medida la señal.

Alguno de los controles que pueden tener los preamplificadores son:
	
\begin{itemize}
	\item Control de volumen
	\item Control de tono
	\item Control de balance
	\item Selector de canal de entrada 
	\item Amplificación
	\end{itemize}	
	
	\medskip 
\subsection{Amplificadores de Potencia}
	
Un amplificador debe satisfacer ciertos requerimientos especiales. Uno de los más importantes es el de entregar una señal con una cantidad específica de potencia a una carga con niveles aceptablemente bajos de distorsión. Otro objetivo común en el diseño es minimizar la impedancia de salida, de tal forma que la ganancia de voltaje quede relativamente poco afectada por el valor de la impedancia de carga. Una etapa de salida bien diseñada debe cumplir con estas características de funcionamiento, consumiendo poca potencia en estado de reposo, sin que esto represente una limitación importante en la respuesta en frecuencia del amplificador. 
 
Los amplificadores de potencia  se clasifican generalmente en seis tipos: A, B, AB , C y G para diseños analógicos y clases D y E para los diseños de conmutación. 
\medskip 


\subsubsection*{Amplificadores Clase A}


En esta clase de amplificadores se usa un solo transistor. El emisor seguidor es la etapa de salida clase A mas utilizada. La corriente de salida circula durante todo el ciclo de la señal de entrada, ya que el transistor esta polarizado con una corriente continua. Esta es una de las grandes desventajas de este tipo de amplificador ya que consume potencia en ausencia de señal y por lo tanto es lógico esperar un rendimiento pobre que en general no supera el 25\%. Como ventaja la distorsión introducida suele ser baja. En la Figura~\ref{ampliA} se muestra un ejemplo de este tipo de amplificador.
 
\begin{figure}[H]
\centering
\includegraphics[scale=0.6]{img/ampliA.png}
\caption{Ejemplo, amplificador clase A}
\label{ampliA} 
\end{figure}

\medskip 
\subsubsection*{Amplificador Clase B}

Esta clase de amplificadores se compone de un par de transistores (uno pnp y otro npn) conectados de forma tal que no se encuentren ambos en la zona de modo activo directo en el mismo instante de tiempo. Es decir, si suponemos tener una entrada senoidal, durante un semiciclo uno de los transistores se encuentra en la región activa, conduciendo corriente, mientras que el otro se encuentra en corte y durante el otro semiciclo viceversa.
 Una ventaja de esta amplificador sobre la clase A, es que los transistores no disipan potencia en ausencia de señal, lo cual mejora la vida util de los transistores y el rendimiento notablemente, alcanzando un máximo del 78\%.
 La desventaja en este tipo de amplificadores es la llamada “distorsión por cruce”. Es fácil detectar su procedencia al analizar la Figura~\ref{ampliB}.

\begin{figure}[H]
\centering
\includegraphics[scale=0.8]{img/ampliB.png}
\caption{Ejemplo, salida clase B.}
\label{ampliB} 
\end{figure}


Se observa que hay un intervalo de tensiones en el cual los transistores no conducen, ese rango generalmente esta dado por $\pm$0.7 V y esta dado por las curvas características de transferencia.

\medskip 
\subsubsection*{Amplificador Clase “AB”}


 Este tipo de amplificadores recurre a la misma topología utilizada en la etapa de salida de los amplificadores clase B, con la salvedad de que aquí en los transistores circulan una corriente de polarización a modo de reducir notablemente la “distorsión por cruce”.
 Existen diferentes formas de logra dicho tipo de polarización. Las mas sencillas implican agregar un resistor o diodos, por los que circula una corriente fija dada por el circuito de polarización o fuente de corriente. La otra forma es utilizar los circuitos conocidos como multiplicadores de VBE , que resulta ser la forma empleada en este trabajo práctico.

\medskip 
\subsubsection*{Amplificador Clase C}


La corriente de salida solo circula durante menos de medio ciclo de la señal de entrada. Y luego se complementa la salida con un circuito compuesto de capacitores e inductores.
La clase C trabaja para una banda de frecuencias estrecha y resulta muy apropiado en equipos de radiofrecuencia. Esto es debido al fenómeno de resonancia el cual se genera a la salida del amplificador cuando es sintonizado (la impedancia capacitiva e inductiva se cancelan a una frecuencia previamente calculada), aunque no trabaja arriba de 180 grados de ciclo, este amplificador a la salida genera una señal de ciclo completo de señal para la frecuencia fundamental. En la Figura~\ref{ampliC} se muestra un ejemplo de una amplificador de esta clase.
No se utiliza en sonido, por su gran nivel de distorsión y por que su operación no esta destinada para amplificadores de gran señal o gran potencia.

\begin{figure}[H]
\centering
\includegraphics[scale=0.35]{img/ampliC.png}
\caption{Ejemplo, amplificador clase C.}
\label{ampliC} 
\end{figure}

\medskip 
\subsubsection*{Amplificador Clase D}

Esta clase de amplificadores usa señales de pulso (digitales). El uso de técnicas digitales hace posible obtener una señal que varía a lo largo del ciclo completo para producir la salida a partir de muchas partes de la señal de entrada. La principal ventaja de la operación en clase D es que los transistores MOSFET de salida trabajan solo en corte y saturación por lo que teóricamente no se disipa potencia en forma de calor y la eficiencia general puede ser muy alta, de entre 90\% a 99\%. En la practica los MOSFETS solo disipan potencia cuando se encuentran conduciendo (saturación) debido a la pequeña resistencia de encendido que poseen, llamada $R_dson$, de todas maneras esta potencia es despreciable ya que $R_dson$ es del orden de las milésimas de ohm. Se utilizan transistores MOSFET ya que son los únicos capaces de conmutar a las elevadas frecuencias de trabajo, del orden de las centenas de KHz llegando a los MHz en algunos casos.

\medskip 
\subsubsection*{Amplificadores Clase G}


Un amplificador clase G funciona conmutando fuentes de alimentación. Para analizar su funcionamiento tendremos en cuenta un circuito básico como se muestra en la Figura~\ref{ampliG}. Mientras el nivel de la señal de entrada sea pequeño (dentro del margen de +/- V1), el amplificador toma la potencia de la fuente V1. Si la señal de entrada excede el nivel de tensión dado por V1, el circuito automáticamente corta el suministro dado por V1 y conmuta a la fuente de alimentación V2 como puede verse en la Figura~\ref{ampliG_salida}. De esta forma la disipación de potencia es compartida por los transistores de salida, logrando así una menor disipación de potencia y una mayor eficiencia.
En la práctica, la clase G se considera linealmente pobre, comparada con la clase B, dado que la conmutación de las fuentes de alimentación se realiza mediante unos diodos, dando de esta manera un resultado alineal, ya que los mismos deben almacenar y desalojar cargas.
 
\begin{figure}[H]
 \centering
 \includegraphics[scale=0.55]{img/ampliG.png}
 \caption{Ejemplo, amplificador clase G.}
 \label{ampliG} 
 \end{figure}
  
\begin{figure}[H]
 \centering
 \includegraphics[scale=0.55]{img/ampliG_salida.png}
 \caption{Encendido de fuentes V2 en salidas clase G.}
 \label{ampliG_salida} 
 \end{figure}

\subsection{Principales Especificaciones de un Amplificador}
\medskip 
\subsubsection*{Potencia Máxima}

Potencia máxima eficaz, o potencia media a régimen continuo es la potencia eléctrica real verificable con instrumentos que puede proporcionar la etapa de salida  a una frecuencia de 1 kHz (frecuencias medias) sobre la impedancia nominal especificada por el fabricante (normalmente 4$\ohm$, 6$\ohm$ u $8\ohm$) y viene dada por la expresión $P_O=  \frac{V_{O(rms)}^2}{Z_O}$. Donde:
\begin{description}
\item $P_O$ es la potencia de salida
\item $V_{O(rms)}$ es la tensión eficaz de salida
\item $Z_O$ es la impedancia de salida
\end{description}

Se especifica la potencia máxima del amplificador en función de una determinada impedancia, generalmente $8\ohm$. Por ejemplo: 100 WRMS sobre 8$\ohm$.
Cabe destacar que si el amplificador es estéreo hay que tener en cuenta si la potencia se refiere a ambos o a cada uno de los canales.
\medskip 
\subsubsection*{Respuesta en Frecuencia}

Es un rango de frecuencias dentro del cual el amplificador responde de igual forma (respuesta plana). Este rango se espera que como mínimo incluya las audiofrecuencias ( 20 a 20kHz).
Pueden especificarse las frecuencias de corte, en donde la potencia cae a la mitad o la tensión de salida cae en 3dB o sino un rango de frecuencias en donde se cumple que la variación en la tensión de salida no supera una cota dado por el fabricante.
\medskip 
\subsubsection*{Rango Dinámico}

El rango dinámico(DR) es el conjunto de valores entre los niveles de mayor y menor salida, en donde el amplificador reproduce fielmente. En general viene especificado en decibeles y en donde el límite superior esta acotado por la distorsión mientras que el menor esta restringido por el ruido de salida. El rango dinámico se calcula con la relación entre ambos limites, de la siguiente forma:

\begin{equation}\label{rango_dinamico_eq}
DR= \frac{S+N}{N}
\end{equation}

donde:
\begin{description}
\item S es la señal máxima permitida
\item N es la señal de ruido
\item DR es el rango dinámico
\end{description}
\medskip 
\subsubsection*{Distorsión Armónica Total}

Si en un sistema no lineal introducimos un tono de frecuencia $f_0$, en la salida tendremos ese mismo tono (con una amplitud y fase posiblemente diferentes) y, sumado a el, otros tonos de frecuencia $2f_0, 3f_0, ...$ llamados armónicos del tono fundamental . Por lo tanto la THD se calcula de la siguiente forma:

\begin{equation}\label{THD_eq}
THD= \frac{\sum Potencia~de ~los ~armonicos}{Potencia~ de ~la ~frecuencia fundamental}=\frac{P_0+P_1+...+P_N}{P_0}
\end{equation}

Es decir, la distorsión armónica es el valor rms de componentes armónicos de la señal de salida, expresadas como un porcentaje rms del fundamental.
Visto de otra forma, la distorsión describe la variación de la forma de onda de la salida del equipo, con respecto a la señal esperada, si el sistema fuese lineal, con respecto a una determinada entrada y se debe básicamente a la alinealidad de los mismos.
\medskip 
\subsubsection*{Distorsión por Intermodulación}


Es la distorsión que se produce cuando dos o mas señales atraviesan simultáneamente un sistema no lineal. Si dos tonos son reproducidos a la vez, pueden interactuar entre sí en el equipo y producir, asimismo, otros nuevos tonos, que son ni más ni menos que la suma y la diferencia de los dos tonos originales (es lo que se conoce como la frecuencia de batido o pulsaciones). Generalmente, los nuevos tonos no son armónicos entre sí ni con los anteriores debido a que la señal salida no es una combinación lineal de la entrada.
\medskip 
\subsubsection*{Distorsión por Intermodulación Transitoria}


Este tipo de distorsión se da principalmente por el retardo que sufre la señal al ser realimentada negativamente. Todo amplificador demora un tiempo entre que la señal de entrada es aplicada y se obtiene la salida correspondiente, llamado tiempo de tránsito. Es decir, cuando utilizamos una realimentación negativa es esperable que al colocar una entrada inmediatamente obtengamos un efecto de la realimentación que afecte a la misma, pero debido a este tiempo de tránsito aparece un efecto no deseado y por lo tanto este tipo de distorsión. Esta altamente relacionada con el slew rate y con el ancho de banda a lazo abierto del sistema. 
\medskip 
\subsubsection*{Slew Rate}
	
Es la máxima pendiente que puede tener la tensión de salida sin sufrir deformaciones. Generalmente se mide en $\frac{V}{\usec}$ y se calcula como:
\begin{equation}
SR = F(max) \times 2\pi \times V_p
\end{equation}
\begin{description}
\item F(max)= Frecuencia máxima de operación
\item  $V_p$= Tensión pico de onda
\end{description}
\medskip 
\subsubsection*{Sensibilidad}

Este parámetro es una relación entre el valor de tensión de entrada que es necesario para producir la máxima potencia de salida y dicha señal de salida.Por lo general se especifica en decibeles a una determinada impedancia. Si la señal de entrada supera el valor especificado por la sensibilidad no existe ninguna garantía que la señal de salida no sufra un recorte que termine dañando algún componente.
\medskip 
\subsubsection*{Relación Señal a Ruido}

La relación señal/ruido se define como el cociente que existe entre la potencia de la señal que se transmite y la potencia del ruido que la corrompe. Este margen es medido en decibeles. A su vez también es importante definir la figura de ruido. La magnitud del ruido generado por un dispositivo electrónico, por ejemplo un amplificador, se puede expresar mediante la denominada figura de ruido (F), que es el resultado de dividir la relación señal/ruido en la entrada $(S/R)_{entrada}$ por la relación señal/ruido en la salida $(S/R)_{salida}$, cuando los valores de señal y ruido se expresan en números simples :

\begin{equation}
F=\frac{(S/R)_{salida}}{(S/R)_{entrada}}
\end{equation}
\medskip 
\subsubsection*{Impedancia de Entrada}


 Es la impedancia equivalente que vería un generador aplicado a la entrada del amplificador. Para el caso particular de este tipo de amplificador (de tensión) buscamos que sea relativamente alta y no cargue a la etapa anterior. Claramente depende de la frecuencia de operación pero un valor típico para el rango de audiofrecuencias es de 10K $\ohm$.
\medskip 
\subsubsection*{Impedancia de Salida}


Es la impedancia equivalente que vería un generador aplicado a la salida del amplificador. En el caso particular del amplificador de audio buscamos que sea muy baja dado que las cargas son relativamente bajas y de lo contrario nos acortarían la amplitud de la señal de salida. Claramente depende de la frecuencia de operación pero un valor típico para el rango de audiofrecuencias es de décimas o centésimas de $\ohm$.
\medskip 
\subsubsection*{Factor de Amortiguamiento}

Indica la relación entre la impedancia nominal del parlante a conectar y la impedancia de salida del amplificador. Un factor de amortiguamiento alto permite mayor control del movimiento de los 
altavoces (evita oscilaciones) y por tanto reduce la distorsión, especialmente en graves. 
\bigskip
\subsection{Fuentes de Alimentación}

\subsubsection*{Fuentes Lineales}

Este tipo de fuentes tienen un diseño relativamente simple, que puede llegar a ser más complejo cuanto mayor es la corriente que deben suministrar, en lineas generales siguen el esquema de la Figura~\ref{fuente_lineal_tipo}.

\begin{figure}[H]
\centering
\includegraphics[width=0.8\textwidth]{img/fuente_lineal_tipo.png}
\caption{Esquema fuente lineal típica.}
\label{fuente_lineal_tipo} 
\end{figure}

En primer lugar el transformador adapta los niveles de tensión y proporciona aislamiento galvánico. El circuito que convierte la corriente alterna en continua se llama rectificador, luego suelen llevar un circuito que disminuye el rizado. La regulación, o estabilización de la tensión a un valor establecido, se consigue con un componente denominado regulador de tensión. La salida puede ser simplemente un capacitor. 

Las ventajas de las fuentes lineales son una mejor regulación, velocidad y buenas características EMC. Y sus principales desventajas son el bajo rendimiento del rectificador y el tamaño del transformador utilizado. 

\subsubsection*{Fuentes Conmutadas}

Una fuente conmutada es un dispositivo electrónico que transforma energía eléctrica mediante transistores en conmutación. Mientras que un regulador de tensión utiliza transistores polarizados en su región activa de amplificación, las fuentes conmutadas utilizan los mismos conmutándolos activamente a altas frecuencias (20-100 kHz típicamente) entre corte y saturación. La forma de onda cuadrada resultante es aplicada a transformadores con núcleo de ferrita (Los núcleos de hierro no son adecuados para estas altas frecuencias) para obtener uno o varios voltajes de salida de corriente alterna que luego son rectificados (Con diodos rápidos) y filtrados para obtener los voltajes de salida de corriente continua. Las ventajas de este método incluyen menor tamaño y peso del núcleo, mayor eficiencia y por lo tanto menor calentamiento. Las desventajas comparándolas con fuentes lineales es que son mas complejas y generan ruido eléctrico de alta frecuencia que debe ser cuidadosamente minimizado para no causar interferencias a equipos próximos a estas fuentes. La Figura~\ref{fuente_conmutada_esquema_bloq} muestra un esquema en bloques de este tipo de fuente.

\begin{figure}[H]
\centering
\includegraphics[width=0.6\textwidth]{img/fuente_conmutada_esquema.jpg}
\caption{Diagrama en bloques fuente conmutada.}
\label{fuente_conmutada_esquema_bloq} 
\end{figure}


La regulación se obtiene con el conmutador, normalmente un circuito PWM (Pulse Width Modulation) que cambia el ciclo de trabajo. Aquí las funciones del transformador son las mismas que para fuentes lineales pero su posición es diferente. El segundo rectificador convierte la señal alterna pulsante que llega del transformador en un valor continuo. La salida puede ser también un filtro de condensador o uno del tipo LC. Las fuentes conmutadas obtienen un mejor rendimiento, menor coste y tamaño comparadas con las lineales.

Existen diversas topologías para este tipo de fuente, aquí solo se mencionarán algunas.

\paragraph*{Topología Flyback: }
Dada su sencillez y bajo costo, es la topología preferida en la mayoría de los convertidores de baja potencia (hasta 100 w). En la Figura~\ref{topo_flyback} se muestran los principios de esta topología de fuente conmutada. Cuando T conduce, la corriente crece linealmente en el primario del transformador. Cuando T se bloquea, el flujo en el transformador cesa generando una corriente inversa en el secundario que carga el condensador a través del diodo alimentando la carga. El condensador mantiene la tensión en la carga durante el período en que T conduce.

\begin{figure}[H]
\centering
\includegraphics[width=0.25\textwidth]{img/topo_flyback.png}
\caption{Topología Flyback.}
\label{topo_flyback} 
\end{figure}

La regulación de tensión en la salida se obtiene mediante comparación con una referencia fija, actuando sobre el tiempo de encendido del transistor, por tanto la energía transferida a la salida mantiene la tensión constante independientemente del valor de la carga o del valor de la tensión de entrada. Los estados del transistor se controlan por modulación de ancho de pulso (PWM) a frecuencia fija. Como se observa en la Figura~\ref{topo_flyback_mul}, esta topología puede implementarse con múltiples bobinados secundarios de manera tal de proveer de manera independiente varias tensiones.

\begin{figure}[H]
\centering
\includegraphics[width=0.32\textwidth]{img/topo_flyback_mul.png}
\caption{Topología flyback con salidas multiples.}
\label{topo_flyback_mul} 
\end{figure}

\paragraph*{Topología Forward: }
Como se ve en la Figura~\ref{topo_forward} es algo más complejo que el sistema Flyback pero rentable en cuanto a costes para potencias de 100 a 250w.Cuando el transistor está conduciendo, la corriente crece en el primario del transformador transfiriendo energía al secundario. La corriente pasa a través de la inductancia L a la carga, acumulándose energía magnética en L.Cuando T se apaga, la corriente en el primario cesa invirtiendo la tensión en el secundario. En este momento D2 queda polarizado inversamente bloqueando la corriente de secundario, pero D3 conduce permitiendo que la energía almacenada en L se descargue alimentando a la carga. El tercer devanado permite aprovechar la energía que queda en el transformador devolviéndola a la entrada, vía D1.

Contrariamente al método Flyback, la inductancia cede energía a la carga todo el tiempo, esto hace que los diodos soporten la mitad de la corriente y los niveles de rizado de salida sean más bajos.

\begin{figure}[H]
\centering
\includegraphics[width=0.35\textwidth]{img/topo_forward.png}
\caption{Topología Forward.}
\label{topo_forward} 
\end{figure}


\paragraph*{Topología Push-Pull: }

Esta topología mostrada en la Figura~\ref{topo_pull-push},  se desarrolló para aprovechar mejor los núcleos magnéticos. En esencia consisten en dos convertidores Forward controlados por dos entradas en contrafase. Los diodos D1 y D2 en el secundario, actúan como dos diodos de recuperación. Idealmente los períodos de conducción de los transistores deben ser iguales, el transformador se excita simétricamente y al contrario de la topología Forward no es preciso prever entrehierro en el circuito magnético, ya que no existe asimetría en el flujo magnético y por tanto componente continua. Ello se traduce en una reducción del volumen del núcleo del orden del 50\% para una misma potencia.

\begin{figure}[H]
\centering
\includegraphics[width=0.35\textwidth]{img/topo_pull-push.png}
\caption{Topología Push-Pull.}
\label{topo_pull-push} 
\end{figure}



% El desarrollo del trabajo fue encarado como un caso real de la vida profesional, en el cual se nos han dado las especificaciones y basamos en ellas nuestro diseño, tratando de ser lo mas eficientes al menor costo posible y con los productos que se pudieron encontrar en el mercado.
%	
%Durante el desarrollo del trabajo hemos ido encontrando inconvenientes, ya sea errores humanos o diferencias entre las simulaciones y la implementación material. Se detallaron dichos problemas ya que consideramos que contribuyen al proceso de aprendizaje del diseño real.
\newpage
\section{Objetivos}
\bigskip
El proyecto consiste en el diseño e implementación de un amplificador de audio que cumpla con las siguientes especificaciones.

\medskip 
\paragraph{Especificaciones iniciales (típicas) de diseño:}

\begin{itemize}
\item Potencia de Salida: desde 25 W a 100 W RMS @ 8 $\ohm$

\item Salida Clase G

\item  Distorsión amónica total(THD): $<$ 0.002 \% a 1 kHz ,$<$ 0.01\% a 10 kHz: 20W (Baja tensión)
\item  Distorsión amónica total(THD): $<$0.003 \% a 1 kHz , $<$ 0.02\% a 10 kHz: 50W (Alta tensión)
\item Respuesta en frecuencia: +/-0.1 dB, 10 Hz – 30 kHz
\item SNR: $<$ -85 dB (20 Hz – 20 kHz)
\item Offset DC: $<$ +/-25 mV
\item Impedancia de entrada: 10 kohm
\item Sensibilidad: 1V RMS
\item Protección por cortocircuito y sobrecarga a la salida
\item Alimentación: 220 VAC +10/-20\%, 50 Hz

\begin{itemize}
    \item Alta tensión: $\sim$ +/-35V a +/-50V (Fuente conmutada)
    \item Baja tensión: $\sim$ +/-20V a +/-25V (Fuente lineal)
\end{itemize}
    
\item  Eficiencia:$>$70\%

\end{itemize}
\medskip 
\paragraph*{Características opcionales:}

\begin{itemize}
\item  Control de volumen VCA
\item  Boost +10 dB @ 30 Hz
\item  Ecualizador gráfico 5 bandas: $+/-$12 dB @64Hz, 250Hz, 1kHz, 4kHz, 12kHz
\item  Modulador / Demodulador FM para Public Adress

\end{itemize}
\newpage

\section{Desarrollo}
{
	\bigskip
\subsubsection{Realimentación}



\bigskip
\subsubsection{Compensación y Slew Rate}

El principal inconveniente al compensar el circuito por polo dominantes es que al agregar un capacitor, este modifica el ancho de banda de potencia. Esto se debe al tiempo que le toma a la etapa anterior cargar el capacitor. Debido a esto la elección del valor de este capacitor debe tener en cuenta ambos efectos y buscar una relación de compromiso entre ambos.

	\bigskip 
\begin{figure}[hbtp]
\centering
\centerline{\includegraphics[scale=0.4]{img/esquema_fuente_lineal.png}}
\caption{Esquema de la fuente lineal}
\end{figure}

	
	\subsection{Simulaciones}
\bigskip
\subsubsection{Polarización}

\begin{figure}[H]
\centering
\centerline{
\includegraphics[width=\textwidth]{img/polarizacion.png}}
\caption{Tensiones de polarización.}
\label{polarizacion_sim} 
\end{figure}
\medskip
\subsubsection{Respuesta en Frecuencia}

Se realizó un barrido en frecuencias de la ganancia del circuito a lazo cerrado para poder observar el ancho de banda del mismo. Como resultado se obtuvo una ganancia de 27.19dB y y se mantiene en el mismo con un error de $\pm$0.1dB entre 5Hz y 66.55kHz.
Estos resultados se pueden observar en la Figura~\ref{resp_frec}, por lo tanto el diseño cumple el requerimiento de banda plana en las frecuencias utilizadas. 

\begin{figure}[H]
\centering
\centerline{
\includegraphics[width=\textwidth]{img/ancho_de_banda.png}}
\caption{Respuesta en frecuencia.}
\label{resp_frec} 
\end{figure}

\bigskip
\subsubsection{Slew Rate}

Para esta simulación se utilizo el circuito de la Figura~\ref{cir_simul_slew_rate}, en el cual la entrada al amplificador es una señal escalón. Se simuló y se tomaron las tensiones en dos puntos, luego se aproxima el slew rate como la pendiente entre estos puntos. Como se ve en la Figura~\ref{simul_slew_rate} con los puntos elegidos se obtuvo un slew rate de $37~ \frac{\volt}{\usec}$.

\begin{figure}[H]
\centering
\includegraphics[width=1\textwidth]{img/slew_rate_cir.png}
\caption{Circuito utilizado para obtener slew rate.}
\label{cir_simul_slew_rate}
\end{figure}

\begin{figure}[H]
\centering
\centerline{\includegraphics[width=1\textwidth]{img/slew_rate.png}}
\caption{Simulación del slew rate.}
\label{simul_slew_rate}
\end{figure}

\medskip
\subsubsection{Estabilidad}

Para obtener el margen de ganancia y fase del circuito se simuló la respuesta en frecuencia de la ganancia a lazo abierto(T), para esto se modificó la topologia del circuito como se ve en la Figura~\ref{cir_simul_estab}. Para obtener el margen de ganancia se determino la ganancia con un angulo de -180º, dando un margen de 3.9dB. Por otro lado, el margen de fase resulto de 72º,siendo la diferencia entre la fase a 0dB y -180º.
Los resultados de la simulación se observan en la Figura~\ref{simul_estab}.

\begin{figure}[H]
\centering
\includegraphics[width=1\textwidth]{img/margen_fase_ganancia_cir.png}
\caption{Circuito utilizado en análisis de estabilidad.}
\label{cir_simul_estab}
\end{figure}


\begin{figure}[H]
\centering
\includegraphics[width=1\textwidth]{img/margen_fase_ganancia.png}
\caption{Respuesta en frecuencia de T.}
\label{simul_estab}
\end{figure}

\medskip
\subsubsection{Eficiencia}

\begin{figure}[H]
\centering
\includegraphics[width=1\textwidth]{img/eficiencia.png}
\caption{Eficiencia del amplificador en función a la potencia disipada en la carga.}
\label{simul_efi}
\end{figure}

\subsubsection{Impedancia de entrada}

Para simular la impedancia de entrada simplemente se puso una fuente de señal a la entrada y se dividió por la corriente que pasaba por ella. Mostramos ahora los resultados de esa simulación.

\begin{figure}[H]
\centering
\includegraphics[width=1\textwidth]{img/Rin.png}
\caption{Resistencia de entrada en función de la frecuencia.}
\label{Rin_sim}
\end{figure}

\subsubsection{Impedancia de salida}

La impedacia de salida se la simuló de dos formas. Una aplicando Thévenin y pasivando la entrada, poniendo una fuente de señal a la salida desacoplada por un capacitor, como se ve en la figura.

\begin{figure}[H]
\centering
\includegraphics[width=1\textwidth]{img/Rout_circ_1.png}
\caption{Circuito para la simulación de la impedancia de salida.}
\label{Rout_sim_circ}
\end{figure}

Luego se dividió la corriente que pasaba por ella por la tensión y tenemos la impedancia vista por la entrada en función de la frecuencia.

\begin{figure}[H]
\centering
\includegraphics[width=1\textwidth]{img/Rout_1.png}
\caption{Simulación de la impedancia de salida.}
\label{Rout_sim}
\end{figure}

La otra simulación que se hizo fue la de la misma medición. Poniendo carga infinita y de $8\ohm$, luego aproximando con $R_{out}=8\ohm(\frac{V_{\inf}}{V_{8\ohm}}-1)$.

\begin{figure}[H]
\centering
\includegraphics[width=1\textwidth]{img/Rout_circ_2.png}
\caption{Circuito para la medición de la impedancia de salida.}
\label{Rout_med_circ}
\end{figure}

El resultado obtenido a $1kHz$ que fue la frecuencia a la que se lo midió, se puede ver en la figura. El resultado de la aproximación es de $18m\ohm$.

\begin{figure}[H]
\centering
\includegraphics[width=1\textwidth]{img/Rout_2.png}
\caption{Simulación de la medición de la impedancia de salida.}
\label{Rout_med}
\end{figure}

	\subsection{Realización del Circuito Impreso}
\bigskip 
\subsubsection{Criterios de Diseño}
Para obtener un circuito impreso que tenga un buen rechazo de ruido y distorsión, hay que cuidar algunas reglas de diseño.
\begin{itemize}
\bigskip 

\item  Caminos de los conductores de alimentación suficientemente anchos y  dispuestos uno próximo al otro, con el objetivo de disminuir el área efectiva y por lo tanto la impedancia.

\item Capacitores de desacople del valor adecuado, de modo que funcionen a la frecuencia correspondiente.

\item Líneas de señal generando la menor área compatible con la distribución de los elementos con su camino de retorno. Especialmente los caminos de alta corriente y/o velocidad como para líneas de gran sensibilidad.

\item Área efectiva del circuito lo más pequeña posible.

\item Conexiones de masas y alimentación sin bucles.

\item Capacidades parásitas entre masa y las líneas de señal minimizadas al alejar pistas.

\item Masas de entrada y salida unidas a un solo punto en común.

\item Disipadores en el borde de la placa para facilitar instalación y optimizar su disipación.

\end{itemize}
\subsubsection{Desarrollo de los Criterios}

Ahora vamos a mostrar como fue dibujado el PCB. Son indicadas las entradas y la salida de señal y los bornes de la alimentación. %Aca se puede ver el esquema entero.\\

%\begin{figure}[H]
%\centering
%\includegraphics[width=\textwidth]{img/PCB1.png}
%\caption{Circuito impreso del amplificador.}
%\end{figure}

\subsubsection*{Reglas generales de dibujo}

El circuito tiene una área efectiva lo más pequeña posible con pistas tal que se minimizan las capacidades parásitas entre masa y las pistas de señal.
Queremos que las conexiones de masa y alimentación no tengan bucles. Para ello, intentamos de evitar los bucles y las pistas paralelas de mucha longitud y la cercanía entre ellas.
Las líneas de señal encierran la menor área posible compatible con la distribución de los elementos en su camino de retorno, cuidando especialmente los caminos de alta corriente y/o tensión de las líneas de gran sensibilidad.
Para disminuir el ruido hemos intentado evitar los puentes en la medida de lo posible, reduciéndolos a 3.

\subsubsection*{Repartición general}
La geometria del circuito respeta lo mejor posible las etapas originales del amplificador de potencia.  Esta disposición permite progeter la señal de entrada la cual es una de las mas débiles en tensión. De hecho, cuidamos que la entrada no sea mezclada con otras pistas de mayor corriente como por ejemplo la alimentacion, para evitar la inducción de ruido. Para eso tambien separamos las masas del circuito en dos lazos que se juntan en un punto unico, separando asi el camino de la corriente de alimentación del camino de la señal.
Este dibujo le permite tambien a juntar los transistores que calentan lo que permite de compartir los disipadores.

%\begin{figure}[H]
%\centering
%\includegraphics[width=\textwidth]{img/PCB2.png}
%\caption{Circuito impreso del amplificador.}
%\end{figure}

\subsubsection*{Alimentación}
Caminos de los conductores de alimentación suficientemente anchos y  dispuestos uno próximo al otro, con el objetivo de disminuir el área efectiva y por lo tanto la impedancia. Además, al estar cerca los caminos positivos y los negativos y no atravesar el circuitos, su campo eléctrico no afecta al resto.
También hicimos un plano de masa en estrella, para no concatenar ruido, con una parte dedicada a la entrada y la otra a la salida. Esto permite disminuir el ruido y proteger la señal de entrada.
 Agregamos después los capacitores de desacople del valor adecuado, de modo que funcionen a la frecuencia correspondiente. Lo hacemos lo más cerca del componente alimentado que sea posible.

\subsubsection*{Los resistencias en el emisor de salida}
Estas dos resistencias son de baja R y es importante que no se vean muy alteradas. Para evitar el cambio de temperatura hemos cuidado a que ninguna pista pasa debajo de estas dos resistencias. Además, los caminos que las conectan con la salida son anchos y perfectamente simétricos. De ésta forma, las pistas no sólo incorporan poca resistencia en serie sino que además, la incorporan en igual magnitud, cuestión de no perder la simetría a la salida, y que la degeneración de los transistores de salida sea lo mas simétrica posible.

\begin{figure}[H]
\centering
\includegraphics[scale=1]{img/PCB3.png}
\caption{Circuito impreso del amplificador.}
\end{figure}

\subsubsection{Disipadores}
\bigskip
Para el calculo de los disipadores se utilizo la ley experimental:
$$
   \theta_{ja}=\dfrac{T_{jm}-T_a}{P_D}
$$
$$
	\theta_{ja}=\theta_{jc}+\theta_{cs}+\theta_{sa}
$$

En la cual $\theta_{ja}$ es la resistencia térmica juntura-ambiente. Para cada transistor que maneje altas corrientes se calcula el valor del disipador requerido teniendo en cuenta la potencia disipada y su resistencia térmica. En el caso del transistor del multiplicador Vbe, que requiere estar a la misma temperatura que los de la salida clase B, se ubicará en el mismo disipador para disminuir la diferencia de temperaturas entre ellos.

\subsubsection{Circuito Implementado}
\begin{figure}[H]
\centerline{
\includegraphics[width=1\textwidth]{img/circuito_implementado_todo.png}}
\caption{Circuito impreso del amplificador.}
\end{figure}

\subsubsection{Fuente Lineal}
\medskip
Para este circuito se utilizaron pistas de 4mm de ancho. Los diodos utilizados en el puente son 6A10 los cuales pueden soportar las corrientes requeridas por el amplificador, ya que soportan hasta 6A; y poseen una caída de tension en directa menor a 1V.
En la Figura~\ref{circuito_impreso_fuente_lineal} se muestra el circuito impreso implementado. 

\begin{figure}[H]
\centering
\centerline{\includegraphics[width=1\textwidth]{img/circuito_impreso_fuente_lineal.png}}
\caption{Circuito impreso de la fuente lineal.}
\label{circuito_impreso_fuente_lineal} 
\end{figure}
\medskip
\subsubsection{Preamplificador}

En este impreso se debió tener en cuenta las posiciones y sentido de giro de los potenciometros para lograr un frente coherente y ordenado. Se agrego un conector jack a la salida para facilitar la desconexión con el amplificador de potencia de ser necesario.
Se utilizaron amplificadores operacionales NE5532, típicos en este tipo de aplicaciones debido a sus buenas prestaciones y bajo ruido.

\begin{figure}[H]
\centering
\centerline{\includegraphics[width=1\textwidth]{img/pre_pcb.png}}
\caption{Circuito impreso del preamplificador.}
\label{pre_pcb} 
\end{figure}

	\subsection{Mediciones}
\bigskip
\subsubsection{Polarización}
Al medir la polarización del circuito se regulo el multiplicador Vbe de forma tal de obtener corrientes similares a la salida. En el Cuadro~\ref{polarizacion} se observan las mediciones que se tomaron para verificar la correcta polarización del circuito.


\begin{table}[H]
\begin{center}
\label{polarizacion} 
\begin{tabular}{|c|c|}
\hline 
\textbf{Resistor} & \textbf{Tensión entre bornes} \\ 
\hline 
RL & $59.2\mvolt$ \\ 
\hline 
R17 & $14.2\mvolt$ \\ 
\hline 
R18 & $15.8\mvolt$ \\ 
\hline 
R10 & 137.8mV \\ 
\hline
R5 & 21.3mV \\ 
\hline
R2 & 20.8mV \\ 
\hline
R4 & 4.187V \\ 
\hline
R8 & 659.4mV \\ 
\hline
\end{tabular} 
\end{center}
\end{table}

\medskip
\subsubsection{Ganancia}
Con una entrada senoidal de 1kHz cuya amplitud se fue variando y registrando las tensiones de salidas sobre la carga de $8\ohm$ se realizo el Cuadro~\ref{ganancia}. De estas mediciones se obtiene que la ganancia es de 23 veces$\simeq$ 27.2dB. Por otro lado, la última medición confirma que se cumple el requerimiento de potencia a la salida, ya que se obtienen 65.6W RMS.


\begin{table}[H]
\begin{center}
\label{ganancia} 
\begin{tabular}{|c|c|}
\hline 
\textbf{Vin(pico)} & \textbf{Vout(pico)} \\ 
\hline 
200mV & 4.6V \\ 
\hline 
500mV & 11.5 \\ 
\hline 
1 & 23V \\ 
\hline 
1.41V1 & 32.4V \\ 
\hline
\end{tabular} 
\end{center}
\end{table}

\medskip
\subsubsection{Respuesta en Frecuencia}
Obtencion de banda de frecuencia en que la ganancia se mantiene constante con un error de 0.1dB.
Estas mediciones se realizaron ingresando una señal senoidal, de amplitud tal, que sobre la carga hubiese 2V pico. Luego se buscaron frecuencias donde la ganancia decreciera 0.1dB, esto es, tension pico de 1.97V a la salida.

\begin{itemize}
\item Frecuencia inferior: $7.7\hertz$
\item Frecuencia superior: $85\khertz$
\end{itemize}


\medskip
\subsubsection{Impedancia de Entrada}

Para realizar esta medición se aplico senoidal de 1kHz y de amplitud tal que a la salida del amplificador hubiese una de 6V pico. Luego se agrego en serie con la entrada una resistencia de 4.7$\kohm$ y un potenciómetro de 10$\kohm$, al que se fue variando hasta que la salida mostrara 3V pico. Por lo tanto, la resistencia de entrada del amplificador y la del serie resistor-potenciómetro eran iguales, midiendo esta última se obtuvo una resistencia de entrada de 10.52$\kohm$.
\medskip
\subsubsection{Impedancia de Salida}

Se determino el valor de la impedancia de salida midiendo la tensión de salida dos veces,
una en vacío (Vo) y otra con carga nominal (Vc) y a una frecuencia de 1 KHz. Resultando:


\begin{itemize}
\item Vo = 2,2939 
\item Vc = 2.2750 
\item $R_{carga}$ = 8.4$\ohm$
\end{itemize}

Luego se puede calcular la impedancia de salida (asumiendo que es totalmente resistiva) con la expresión: 
$$
Zo = R_{carga} \times (\frac{Vo}{Vc} -1) \simeq 0.07 \ohm
$$
\medskip
\subsubsection{Slew Rate}

Para obtener elvalor del Slew Rate del circuito se procedió a ingresar una señal rectangular de 1kHz al amplificador y observar la salida del mismo. Los resultados se muestran en la Figura~\ref{slew_rate_completo}.

\begin{figure}[H]
\centering
\includegraphics[width=0.9\textwidth]{img/slew_rate_salida.jpg}
\caption{Resultados Slew Rate}
\label{slew_rate_completo} 
\end{figure}


Por lo tanto el Slew Rate se calcula:
$$
Slew Rate =\frac{10V \times 4.4}{6 \times 500ns} = 14.67 \frac{V}{\usec}
$$
\medskip
\subsubsection{Factor de amortiguamiento}

Habiendo medido previamente la impedancia de salida, se calcular el factor de amortiguamiento
como: 
$$
FA = \frac{R_{CARGA NOMINAL}}{Z_{SALIDA}} = 114,28
$$
El factor de amortiguamiento será diferente a distintas frecuencias, el que aqui se calcula es el de 1kHz, debido a que a esa frecuencia se midió la impedancia de salida.

\subsubsection{Distorsión}

Para medir distorsión se utilizó el software Spectralab en conjunto con una placa externa de sonido de baja distorsión, alrededor del $0.003\%$, tanto como para regular la amplitud de la señal de entrada como para recibir la señal de salida y poder analizarla en la computadora.
Dado que la placa de sonido contaba con una entrada máxima de $1V_{rms}$, no podría tomar directamente la señal de salida, por lo cual se construyeron dos divisores de tensión. El primero fue para tratar la salida de la señal de $0.3V_{rms}$, es decir $V_o=23 \times 0.3V=6.9V$. Se utilizó un divisor resistivo compuesto de tres resistencias en serie: $10\kohm$, $2.2\kohm$ y $4.7\kohm$;se midio sobre la de $2.2\kohm$ para atenuar 7.68 veces y obtener $0.89V_{rms}$ a la salida. Idénticamente con la señal de $1V_{rms}$, se atenuó de $V_o=23\times1V=23V_{rms}$ a $0.87V_{rms}$ con un divisor formado por unas resistencias de $120\kohm$ y  $4.7\kohm$.
De esta forma se fueron realizando las mediciones de distorsión armónica y distorsión por intermodulación, a distintas frecuencias y amplitudes, las cuales se detallaran a continuación.

\subsubsection*{THD}

Para medir THD se generó una onda senoidal de frecuencia $1kHz$ y $7kHz$ de baja tensión, $0.3V_{rms}$ y de alta tensión, $1V_{rms}$, primero con salida abierta y luego con una carga de $8\ohm$. Graficamos ahora los resultados:

\begin{figure}[H]
\centering
\includegraphics[width=\textwidth]{img/Distorsion/THD_1k_03V_span_20_4520.png}
\caption{THD a 1kHz, $0.3V_{rms}$, span 20-4520Hz, sin carga.}
\label{THD1} 
\end{figure}

\begin{figure}[H]
\centering
\includegraphics[width=\textwidth]{img/Distorsion/THD_7k_03V_span_4750_9250.png}
\caption{THD a 7kHz, $0.3V_{rms}$, span 4750-9250Hz, sin carga.}
\label{THD2} 
\end{figure}

\begin{figure}[H]
\centering
\includegraphics[width=\textwidth]{img/Distorsion/THD_1k_1V_span_20_4520.png}
\caption{THD a 7kHz, $1V_{rms}$, span 20-4520Hz, sin carga.}
\label{THD3} 
\end{figure}

\begin{figure}[H]
\centering
\includegraphics[width=\textwidth]{img/Distorsion/THD_7k_1V_span_4750_9250.png}
\caption{THD a 7kHz, $1V_{rms}$, span 4750-9250Hz, sin carga.}
\label{THD4} 
\end{figure}

Luego se realizó las mismas mediciones pero con carga.

\begin{figure}[H]
\centering
\includegraphics[width=\textwidth]{img/Distorsion/THD_1k_03V_carga_span_20_4520.png}
\caption{THD a 1kHz, $0.3V_{rms}$, span 20-4520Hz, con carga.}
\label{THD5} 
\end{figure}

\begin{figure}[H]
\centering
\includegraphics[width=\textwidth]{img/Distorsion/THD_7k_03V_carga_span_4750_9250.png}
\caption{THD a 7kHz, $0.3V_{rms}$, span 4750-9250Hz, con carga.}
\label{THD6} 
\end{figure}

\begin{figure}[H]
\centering
\includegraphics[width=\textwidth]{img/Distorsion/THD_1k_1V_carga_span_20_4520.png}
\caption{THD a 7kHz, $1V_{rms}$, span 20-4520Hz, con carga.}
\label{THD7} 
\end{figure}

\begin{figure}[H]
\centering
\includegraphics[width=\textwidth]{img/Distorsion/THD_7k_1V_carga_span_4750_9250.png}
\caption{THD a 7kHz, $1V_{rms}$, span 4750-9250Hz, con carga.}
\label{THD8} 
\end{figure}

\subsubsection*{IMD}

	\subsection{Comparativa Mediciones-Simulaciones}
\bigskip

	\subsection{Errores y Modificaciones al Diseño Original}
\bigskip
\subsubsection{Protecciones contra cortocircuitos}
\medskip
Como se puede ver en la sección \ref{contra_cortos} se diseñaron las protecciones utilizando una resistencia en el emisor de los transistores para definir la corriente a la cual estos conducirían. Pero esto es una falla ya que con este diseño los limites de corriente empezarian a depender de la corriente de emisor de las protecciones, disminuyendo la eficiencia de las mismas. 
}
\newpage
\section{Conclusiones}

A lo largo de este extenso proyecto hemos tratado de incorporar todos los conocimientos aprendidos en la cursada de la materia, corroborado una vez mas en cuanto puede diferir teoría de práctica. Se ha logrado cumplir con el objetivo planteado, salvo por ciertas especificaciones que estuvimos cercanas alcanzar.

La manipulación de señales de audio resultó ser susceptible a muchos parámetros externos que en otros tipos amplificadores no se suele ser cuidadoso. De esta manera hemos aprendido sobre las diferentes formas de distorsión y sobre sus causas, en la alinealidad en un amplificador.
Otra dificultad con la que nos encontramos fue la alta potencia, la cual nos obligó a ser más cuidadosos con el diseño y la construcción, ademas de hacer evidente efectos adversos y propiedades que hasta esta altura de la carrera, habíamos despreciado, por ejemplo, la disipación de calor.

Finalmente, aunque el proyecto presentado fue complicado de realizar y cumple con la mayoría de las especificaciones, sigue siendo un modelo básico y primitivo, susceptible a muchas mejoras.

Para terminar cabe destacar que el trabajo en grupo fue decisivo para la realización del proyecto. Siendo éste algo que se desarrolla con frecuencia dentro del ámbito tanto de la facultad como el laboral, lo consideramos central para el desarrollo de un ingeniero y creemos que fue adquirido por todos los miembros del equipo.


\bigskip
\newpage
\section{Anexos}
\bigskip
\subsection{TL494}\label{TL494}

El TL494 es un circuito de control de modulación de ancho de pulso a frecuencia constante. La modulación del pulso de salida es lograda comparando la forma de onda triangular entregada por el oscilador interno con cualquiera de las señales de control. Estas señales pueden ser sacadas del circuito del control de tiempo muerto (deadtime control) o del amplificador de error. La entrada de control de tiempo muerto es comparada directamente con una compensación de 120mV (que fija un tiempo muerto por default). El comparador PWM compara la señal de control creada por los amplificadores de error con la triangular. La función del amplificador de error es supervisar la tensión de salida y proporcionar la ganancia suficiente de modo que unos pocos milivolts de error en su entrada causen una señal de control de amplitud suficiente para proporcionar control de modulación del $100\% $. Los amplificadores de error también pueden ser usados para supervisar la corriente de salida y proporcionar la limitación de corriente con la carga.

\subsubsection{Terminales del TL494}

Se procede a detallar las terminales de este integrado tal como se muestran en la Figura~\ref{TL494}.

\begin{figure}[H]
\centering
\includegraphics[width=\textwidth]{img/tl494.png}
\caption{Diagrama en bloques del circuito interno del integrado TL494.}
\label{tl494}
\end{figure}

\begin{enumerate}
\item Terminal positivo del comparador 1, utilizado para la realimentación.
\item Terminal negativo del comparador 1, utilizado para la realimentación.
\item Terminal de realimentación del PWM y terminal positivo del comparador PWM. Se utiliza para evitar efectos parásitos.
\item Terminal del circuito de control de tiempo muerto (deadtime control). Se utiliza para evitar efectos parásitos
\item Terminal del oscilador. Para que el oscilador funcione se conecta un capacitor entre terminal y masa.
\item Terminal del oscilador. Para que el oscilador funcione se conecta un resistor entre terminal y masa.
\item Terminal del regulador de referencia. Para que el regulador funcione se conecta este terminal a masa.
\item Terminal de drain del transistor 1. 
\item Terminal de source del transistor 1
\item Terminal de drain del transistor 2
\item Terminal de source del transistor 2
\item Terminal de alimentación del circuito
\item Terminal de salida utilizado como control. Se utiliza para evitar efectos parásitos.
\item  Terminal que brinda la tensión de referencia de 5V del regulador de referencia interno del circuito.
\item Terminal positivo del comparador 2, utilizado para la realimentación.
\item Terminal negativo del comparador 2, utilizado para la realimentación.
\end{enumerate}

\subsubsection{Seteo de frecuencia de trabajo}

Dado que se deseaba una frecuencia de 80kHz se utilizó el gráfico de la Figura~\ref{frec_tl494} para determinar el valor del resistor y capacitor en las terminales del TL494.

\begin{figure}[H]
\centering
\includegraphics[width=0.4\textwidth]{img/frec_TL494.png}
\caption{Frecuencias de trabajo del TL494 en función del RC en los terminales .}
\label{frec_tl494}
\end{figure}

\end{document}

