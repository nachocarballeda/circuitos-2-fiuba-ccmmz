\section{Conclusiones}

A lo largo de este extenso proyecto hemos tratado de incorporar todos los conocimientos aprendidos en la cursada de la materia, corroborado una vez mas en cuanto puede diferir teoría de práctica. Se ha logrado cumplir con el objetivo planteado, salvo por ciertas especificaciones que estuvimos cercanas alcanzar.

La manipulación de señales de audio resultó ser susceptible a muchos parámetros externos que en otros tipos amplificadores no se suele ser cuidadoso. De esta manera hemos aprendido sobre las diferentes formas de distorsión y sobre sus causas, en la alinealidad en un amplificador.
Otra dificultad con la que nos encontramos fue la alta potencia, la cual nos obligó a ser más cuidadosos con el diseño y la construcción, ademas de hacer evidente efectos adversos y propiedades que hasta esta altura de la carrera, habíamos despreciado, por ejemplo, la disipación de calor.

Finalmente, aunque el proyecto presentado fue complicado de realizar y cumple con la mayoría de las especificaciones, sigue siendo un modelo básico y primitivo, susceptible a muchas mejoras.

Para terminar cabe destacar que el trabajo en grupo fue decisivo para la realización del proyecto. Siendo éste algo que se desarrolla con frecuencia dentro del ámbito tanto de la facultad como el laboral, lo consideramos central para el desarrollo de un ingeniero y creemos que fue adquirido por todos los miembros del equipo.


\bigskip