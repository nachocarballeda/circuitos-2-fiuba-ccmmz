\subsubsection{Criterios de Diseño}

\begin{itemize}
\bigskip 

\item  Caminos de los conductores de alimentación suficientemente anchos y  dispuestos uno próximo al otro, con el objetivo de disminuir el área efectiva y por lo tanto la impedancia.

\item Capacitores de desacople del valor adecuado, de modo que funcionen a la frecuencia correspondiente.

\item Líneas de señal generando la menor área compatible con la distribución de los elementos con su camino de retorno. Especialmente los caminos de alta corriente y/o velocidad como para líneas de gran sensibilidad.

\item Área efectiva del circuito lo más pequeña posible.

\item Conexiones de masas y alimentación sin bucles.

\item Capacidades parásitas entre masa y las líneas de señal minimizadas al alejar pistas.
\end{itemize}

\subsubsection{Circuito Implementado}
\begin{figure}[hbtp]
\centerline{
\includegraphics[scale=0.55]{img/circuito_implementado_todo.png}}
\caption{Circuito impreso del amplificador}
\end{figure}

\subsubsection{Fuente Lineal}
\begin{figure}[hbtp]
\centering
\centerline{\includegraphics[scale=0.53]{img/circuito_impreso_fuente_lineal.png}}
\caption{Circuito impreso de la fuente lineal}
\end{figure}
