\subsection{Realización del Circuito Impreso}
\bigskip 
\subsubsection{Criterios de Diseño}

A la hora de la implementación de los circuitos, se tomaron en cuenta las siguientes reglas de diseño:
\begin{itemize}
\bigskip 

\item Se cuidó que los caminos de los conductores de alimentación sean suficientemente anchos para reducir la resistividad parásita y que estén dispuestos uno próximo al otro, con el objetivo de disminuir el campo eléctrico generado por ellos.

\item Para el cálculo de los capacitores de desacople tuvimos en cuenta el ancho de banda con el que estábamos trabajando, de modo que funcionen a la frecuencia correspondiente y presenten poca impedancia.

\item Se trató de hacer lo mas cortos y eficientes posibles los recorridos de los caminos de señal, para poder reducir la interferencia con los demás elementos del circuito.

\item Conexiones de masas, alimentación y señal sin lazos cerrados, con el objeto de no concatenar ruido.

\item Las líneas de señal y masa se separaron lo máximo posible para reducir las capacitancias parásitas.

\item Las masas de alimentación y del camino de señal se separaron para que el ruido de la línea no contamine la señal. Solo se unen en un punto.

\item Disipadores en el borde de la placa para facilitar instalación y optimizar la disipación de calor.

\item La señal de realimentación se cuidó de no tomarla inmediatamente del nodo donde convergen las corrientes de salida, sino de un punto que tiene la misma tensión pero no circulan corrientes tan altas, para minimizar el ruido y la distorsión por toma de realimentación.

\end{itemize}
\bigskip
\subsubsection*{Distribución general}
La ubicación de los elementos del circuito respeta lo mejor posible las etapas originales del amplificador de potencia. Ésto facilita el seguimiento visual de los componentes y permite detectar fallas mas rápidamente.
Los caminos de los conductores de alimentación se hicieron los suficientemente anchos y dispuestos uno próximo al otro. Al estar cerca los caminos positivos y los negativos y no rodear el circuito, su campo eléctrico no afecta al resto de los componentes.
También se realizó un plano de masa en forma de estrella, para no concatenar ruido, con una parte dedicada a la entrada y la otra a la salida. Esto permite disminuir el ruido y proteger la señal de entrada.
\bigskip
\subsubsection*{Tecnologia de Materiales}

Donde fue posible se utilizó resistencias de metalfilm de potencias superiores a las eso no solo da un rango mayor de funcionamiento en caso de sobrepasar las condiciones limite sino que minimiza el ruido Flicker que estas podrían agregar.
En los circuitos donde fue necesario se utilizaron placas fotosensibles de manera que el pcb quedara nítido y no hubiera que corregir ninguna pista.

\bigskip
\subsubsection*{Resistencias en el Emisor de Salida}
Estas dos resistencias son de baja R y es importante que no se vean muy alteradas. Para evitar el cambio de temperatura hemos cuidado a que ninguna pista pasa debajo de estas dos resistencias. Además, los caminos que las conectan con la salida son anchos y perfectamente simétricos. De ésta forma, las pistas no sólo incorporan poca resistencia en serie sino que además, la incorporan en igual magnitud, cuestión de no perder la simetría a la salida, y que la degeneración de los transistores de salida sea lo mas simétrica posible.
\bigskip

\subsubsection{Disipadores}

Para cada transistor que maneje altas corrientes se calcula el valor del disipador requerido teniendo en cuenta la potencia disipada y su resistencia térmica. En el caso del transistor del multiplicador $V_{be}$, que requiere estar a la misma temperatura que los de la salida clase B, se ubicará en el mismo disipador para así mantener a todos a la misma temperatura. Los otros dos transistores que deben disipar demasiada potencia se pondrán en sus disipadores propios.Sabiendo que:

$$
   \theta_{ja}=\dfrac{T_{jm}-T_a}{P_D}
$$
$$
	\theta_{ja}=\theta_{jc}+\theta_{cd}+\theta_{da}
$$

En la cual:
\begin{list}{ }
\item $P_D$  es la potencia que debe disipar el transistor.
\item $T_{jm}$ es la temperatura máxima en la juntura.
\item  $T_a$ es la temperatura ambiente.
\item  $\theta_{ja}$ es la resistencia térmica juntura-ambiente.
\item  $\theta_{da}$ es la resistencia térmica disipador-ambiente.
\item  $\theta_{jc}$ es la resistencia térmica juntura-encapsulado.
\item  $\theta_{cd}$ es la resistencia térmica encapsulado-disipador.
\end{list}
 %USE POTENCIA PROMEDIO EN TODOS LADOS!!!
En todos los transistores se utilizo mica y silicona termoconductora, por lo tanto en todos los casos se tiene una $\theta_{cd}=1.5^oC/W$.
Por otro lado en los cuatro transistores se tiene de su hoja de datos: $T_{jm}=150^oC$ y según sus datasheet los MJE350/MJE340 tienen un $\theta_{jc}$ de $0.63^oC/W$, mientras que en los 2sa1302/2sc2381 es de $0.83^oC/W$; la temperatura en el gabinete se supuso de $40^oC$.

Según las simulaciones la potencia disipada en los transistores $Q_{16}$ y $Q_{19}$ es poco menor a 12W. De las ecuaciones anteriores se tiene:
$$
	\theta_{da}=\dfrac{T_{jm}-T_a}{P_D} - \theta_{jc}+\theta_{cd}
$$

Por lo tanto el $\theta_{da}$ para estos transistores debe ser de $6.84^oC/W$. Se adquirieron dos disipadores ZD-6(como se observa en la Figura~\ref{zd6}) de un  $\theta_{da}=2.9^oC/W$ los cuales fueron fáciles de obtener y de una resistencia térmica más que suficiente.

\begin{figure}[H]
\centerline{
\includegraphics[width=0.4\textwidth]{img/zd-6.jpg}}
\caption{Disipador ZD-6.}
\label{zd6} 
\end{figure}

En el caso de $Q_{18}$(que debe disipar 0.5W), $Q_{13}$(que debe disipar 0.5W),$Q_{17}$(que debe disipar 5.33W),$Q_{21}$(que debe disipar 6.35W) y $Q_{11}$(que debe disipar 30mW), los cuales deben estar en el mismo disipador.
Llamando a la resistencia térmica de los MJE350/MJE340 $\theta_{jc1}=0.63^oC/W$ y $\theta_{jc2}=0.83^oC/W$ a las de los 2sa1302/2sc2381. Se puede armar el siguiente circuito térmico:

\begin{figure}[H]
\centerline{
\includegraphics[width=0.6\textwidth]{img/disipador_cir.png}}
\caption{Circuito térmico del disipador de varios transistores.}
\end{figure}

Como las temperaturas de las junturas deben ser iguales, la igualdad de alguans resistencias térmicas y perdiendo la relación de potencias disipadas(luego se supondrá el caso extremo de 7W por transistor), se puede simplificar el circuito, resultando el siguiente circuito:

\begin{figure}[H]
\centerline{
\includegraphics[width=0.4\textwidth]{img/disipador_cir2.png}}
\caption{Circuito térmico simplificado del disipador de varios transistores.}
\end{figure}

Siendo la resistencia equivalente resultante de $0.44^oC/W$, por lo tanto el $\theta_{da}$ para estos transistores debe ser de $2.7^oC/W$. Debido a esto se adquirió un disipador ZD-14(como se observa en la Figura~\ref{zd14}) el cual tiene una $\theta_{da}=1^oC/W$, dando un amplio rango de seguridad.
%el parrafo de arriba supone q todos disipan un maximo de 7W
\begin{figure}[H]
\centerline{
\includegraphics[width=0.4\textwidth]{img/zd-14.jpg}}
\caption{Disipador ZD-14.}
\label{zd14} 
\end{figure}


\bigskip
\subsubsection{Circuito Impreso del Amplificador}

En la Figura 26 se puede observar el circuito impreso realizado para el amplificador. Son indicadas las entradas y la salida de señal y los bornes de la alimentación.
\medskip
\begin{figure}[H]
\centerline{
\includegraphics[width=1\textwidth]{img/PCB1.png}}
\caption{Circuito impreso del amplificador.}
\end{figure}

\bigskip
\subsubsection{Fuente Lineal}
\medskip
Para este circuito se utilizaron pistas de 4mm de ancho. Los diodos utilizados en el puente son 6A10 los cuales pueden soportar las corrientes requeridas por el amplificador, ya que soportan hasta 6A; y poseen una caída de tension en directa menor a 1V.
En la Figura~\ref{circuito_impreso_fuente_lineal} se muestra el circuito impreso implementado. 



\begin{figure}[H]
\centering
\centerline{\includegraphics[width=1\textwidth]{img/circuito_impreso_fuente_lineal.png}}
\caption{Circuito impreso de la fuente lineal.}
\label{circuito_impreso_fuente_lineal} 
\end{figure}
\bigskip
\subsubsection{Preamplificador}
\medskip
En este impreso se debió tener en cuenta las posiciones y sentido de giro de los potenciometros para lograr un frente coherente y ordenado. Se agrego un conector jack a la salida para facilitar la desconexión con el amplificador de potencia de ser necesario.
Se utilizaron amplificadores operacionales NE5532, típicos en este tipo de aplicaciones debido a sus buenas prestaciones y bajo ruido.

\begin{figure}[H]
\centering
\centerline{\includegraphics[width=1\textwidth]{img/pre_pcb.png}}
\caption{Circuito impreso del preamplificador.}
\label{pre_pcb} 
\end{figure}