\bigskip
El proyecto consiste en el diseño e implementación de un amplificador de audio que cumpla con las siguientes especificaciones.

\medskip 
\paragraph{Especificaciones iniciales (típicas) de diseño:}

\begin{itemize}
\item Potencia de Salida: desde 25 W a 100 W RMS @ 8 $\ohm$

\item Salida Clase G

\item  Distorsión amónica total(THD): $<$ 0.002 \% a 1 kHz ,$<$ 0.01\% a 10 kHz: 20W (Baja tensión)
\item  Distorsión amónica total(THD): $<$0.003 \% a 1 kHz , $<$ 0.02\% a 10 kHz: 50W (Alta tensión)
\item Respuesta en frecuencia: +/-0.1 dB, 10 Hz – 30 kHz
\item SNR: $<$ -85 dB (20 Hz – 20 kHz)
\item Offset DC: $<$ +/-25 mV
\item Impedancia de entrada: 10 kohm
\item Sensibilidad: 1V RMS
\item Protección por cortocircuito y sobrecarga a la salida
\item Alimentación: 220 VAC +10/-20\%,50 Hz
    – Baja tensión: ~ +/-20V a +/-25V (Fuente lineal)
    – Alta tensión: ~ +/-35V a +/-50V (Fuente conmutada)
\item  Eficiencia:$>$70\%

\end{itemize}
\medskip 
\paragraph*{Características opcionales:}

\begin{itemize}
\item  Control de volumen VCA
\item  Boost +10 dB @ 30 Hz
\item  Ecualizador gráfico 5 bandas: $+/-$12 dB @64Hz, 250Hz, 1kHz, 4kHz, 12kHz
\item  Modulador / Demodulador FM para Public Adress

\end{itemize}